В настоящее время в нашей стране крайне актуально замещение продуктов зарубежных
IT-компаний, покинувших рынок. Для того, чтобы разрабатываемые продукты 
пользовались спросом, их качество должно быть высоким. Если пропустить этап 
тестирования продукта, то тестировать его будут сами пользователи. Вероятнее 
всего они столкнутся со многими трудностями во время использования и потратят 
много времени на решение проблем с помощью службы поддержки. Всё это может 
привести к отказу от пользования продукта. Таким образом, тестировщики, которые 
помогают улучшить качество разрабатываемого продукта, всегда будут востребованы 
компаниями.

Целью данной работы является изучение основной информации о профессии 
тестировщик.

Поставленная цель достигается исследованием следующих задач:
\begin{enumerate}
    \item Изучить роль тестирования во время работы над проектом
    \item Изучить основные обязанности профессии
    \item Изучить главные инструменты и способы, применяемые при тестировании
\end{enumerate}

В качестве источников информации в реферате используются различные статьи и 
книги по теме реферата.