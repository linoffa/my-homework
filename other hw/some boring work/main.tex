\documentclass[referat]{SCWorks}
% Тип обучения (одно из значений):
%    bachelor   - бакалавриат (по умолчанию)
%    spec       - специальность
%    master     - магистратура
% Форма обучения (одно из значений):
%    och        - очное (по умолчанию)
%    zaoch      - заочное
% Тип работы (одно из значений):
%    coursework - курсовая работа (по умолчанию)
%    referat    - реферат
%  * otchet     - универсальный отчет
%  * nirjournal - журнал НИР
%  * digital    - итоговая работа для цифровой кафдры
%    diploma    - дипломная работа
%    pract      - отчет о научно-исследовательской работе
%    autoref    - автореферат выпускной работы
%    assignment - задание на выпускную квалификационную работу
%    review     - отзыв руководителя
%    critique   - рецензия на выпускную работу
% Включение шрифта
%    times      - включение шрифта Times New Roman (если установлен)
%                 по умолчанию выключен
\usepackage{preamble}

\begin{document}

% Кафедра (в родительном падеже)
\chair{информатики и программирования}

% Тема работы
\title{Тестировщик}

% Курс
\course{1}

% Группа
\group{151}

% Факультет (в родительном падеже) (по умолчанию "факультета КНиИТ")
% \department{факультета КНиИТ}

% Специальность/направление код - наименование
% \napravlenie{02.03.02 "--- Фундаментальная информатика и информационные технологии}
% \napravlenie{02.03.01 "--- Математическое обеспечение и администрирование информационных систем}
% \napravlenie{09.03.01 "--- Информатика и вычислительная техника}
\napravlenie{09.03.04 "--- Программная инженерия}
% \napravlenie{10.05.01 "--- Компьютерная безопасность}

% Для студентки. Для работы студента следующая команда не нужна.
\studenttitle{Студентки}

% Фамилия, имя, отчество в родительном падеже
\author{Фединой Ангелины Ильиничной}

% Заведующий кафедрой 
\chtitle{доцент, к.\,ф.-м.\,н.}
\chname{С.\,В.\,Миронов}

% Руководитель ДПП ПП для цифровой кафедры (перекрывает заведующего кафедры)
% \chpretitle{
%     заведующий кафедрой математических основ информатики и олимпиадного\\
%     программирования на базе МАОУ <<Ф"=Т лицей №1>>
% }
% \chtitle{г. Саратов, к.\,ф.-м.\,н., доцент}
% \chname{Кондратова\, Ю.\,Н.}

% Научный руководитель (для реферата преподаватель проверяющий работу)
\satitle{доцент, к.\,ф.-м.\,н.} %должность, степень, звание
\saname{Г.\,Г.\,Наркайтис}

% Руководитель практики от организации (руководитель для цифровой кафедры)
\patitle{доцент, к.\,ф.-м.\,н.}
\paname{С.\,В.\,Миронов}

% Руководитель НИР
% \nirtitle{доцент, к.\,п.\,н.} % степень, звание
% \nirname{В.\,А.\,Векслер}

% Семестр (только для практики, для остальных типов работ не используется)
\term{2}

% Наименование практики (только для практики, для остальных типов работ не
% используется)
\practtype{учебная}

% Продолжительность практики (количество недель) (только для практики, для
% остальных типов работ не используется)
\duration{2}

% Даты начала и окончания практики (только для практики, для остальных типов
% работ не используется)
\practStart{01.07.2022}
\practFinish{13.01.2023}

% Год выполнения отчета
\date{2023}

\maketitle

% Включение нумерации рисунков, формул и таблиц по разделам (по умолчанию -
% нумерация сквозная) (допускается оба вида нумерации)
\secNumbering

\tableofcontents

% Раздел "Обозначения и сокращения". Может отсутствовать в работе
% \abbreviations
% \begin{description}
%     \item ... "--- ...
%     \item ... "--- ...
% \end{description}

% Раздел "Определения". Может отсутствовать в работе
% \definitions

% Раздел "Определения, обозначения и сокращения". Может отсутствовать в работе.
% Если присутствует, то заменяет собой разделы "Обозначения и сокращения" и
% "Определения"
% \defabbr

\intro
\input{intro.tex}
% После введения — серии \section, \subsection и т.д.
\section{ОБЯЗАННОСТИ ТЕСТИРОВЩИКА}
\subsection{РОЛЬ ТЕСТИРОВЩИКА В РАЗРАБОТКЕ}
Для любой IT-компании, разрабатывающей и поставляющей на рынок свой продукт, 
тестирование является важным и ценным этапом жизненного цикла разработки 
программного обеспечения (ПО). Если процесс тестирования игнорируется то, может 
пострадать как сам продукт, так и компания разработчика в целом, поскольку 
ошибки в ПО могут быть дорогостоящими или даже опасными.

Рассмотрим, как именно тестировщики помогают достичь успеха проекту. Тесное 
сотрудничество тестировщиков с разработчиками, улучшает понимание тестировщиками
каждой части кода и позволяет сконцентрироваться на тех областях, где наиболее 
высок риск возникновения ошибок. В свою очередь, обмен данными полезен и 
разработчикам для наиболее быстрого исправления неполадок. Таким образом, 
взаимодействие между тестировщиками и разработчиками позволяет упростить 
контроль качества и сократить сроки разработки продукта, снизить стоимость 
внесения исправлений \cite{Import_test}.

Кроме того, тестировщики многократно проверяют программное обеспечение перед его 
выпуском. Это помогает в обнаружении ошибок, которые иначе могли бы остаться 
незамеченными.

Поставка ПО хорошего качества, обладающего уникальными функциями, всегда была 
приоритетом в индустрии программного обеспечения. Поэтому участие тестировщика в 
разработке является очень важным.  

\subsection{КВАЛИФИКАЦИЯ И НАВЫКИ ТЕСТИРОВЩИКА}
Несомненно, в профессии важны личные качества человека. Рассмотрим, какие 
являются наиболее важными \cite{SoftSkills}:
\begin{enumerate}
    \item Усидчивость и терпение. Тестировщику необходимо терпеливо искать 
    дефекты в ПО, отслеживая даже самые малейшие отклонения.
    \item Нестандартное мышление. Практически каждый из нас, включая детей и 
    пенсионеров, посещает и использует самые разные сайты компаний ежедневно. 
    Поэтому тестировщик должен уметь представлять самые разнообразные случаи, 
    которые могут возникнуть при взаимодействии клиента с ПО.
    \item Наблюдательность. Необходима, чтобы не пропустить даже самые 
    малейшие отклонения, ведь в дальнейшем это может сильно сказаться на 
    качестве продукта.
    \item Ответственность. Это качество важно тестировщику для того, чтобы он 
    постоянно стремился сделать продукт лучше.
    \item Коммуникабельность и умение работать в команде. Тестировщику для 
    успешной разработки приходится постоянно взаимодействовать с разработчиками, 
    дизайнерами и другими участниками этого процесса.
\end{enumerate}
Кроме того, от тестировщика требуются и навыки такие как:
\begin{enumerate}
    \item Знание основ тестирования, его видов и методов;
    \item Умение составлять тест-кейсы;
    \item Знание языка запросов SQL, умение работать с базами данных;
    \item Знание языков программирования;
    \item Знание систем контроля версий, например, Git.
\end{enumerate}
А также немаловажно владение инструментами ручного и автоматического 
тестирования. Это могут быть:
\begin{enumerate}
    \item Cистемы для создания тест-кейсов и отслеживания ошибок.
    \item Файловые менеджеры, текстовые и XML-редакторы.
    \item Генераторы тестовых данных и другие.
\end{enumerate}
Всё сказанное сочетает в себе хороший тестировщик, помогающий развивать продукт 
в лучшую сторону.
\subsection{ПРОФЕССИЯ «ТЕСТИРОВЩИК» СЕГОДНЯ}
В настоящее время в российских вузах нет отдельных направлений для 
инженеров"=тестировщиков, поэтому начать осваивать данную профессию можно 
самостоятельно или же можно обратиться к набирающему популярность методу 
обучения "--- онлайн"=курсам. Например, «Яндекс Практикум» предлагает пройти 
курс «Инженер по тестированию» и за 4 месяца приобрести навыки, необходимые для 
дальнейшего развития в этой области \cite{Yandex}. 

Условия работы тестировщика имеют довольно большую гибкость. В зависимости от 
конкретной вакансии можно работать в офисе, удалённо, или вовсе на фрилансе.

Немаловажным показателем любой профессии является её уровень заработной платы. 
Так, зарплата специалиста по тестированию зависит преимущественно от его опыта, 
навыков и работодателя, на которого он работает.

Исходя из статистики, медианная заработная плата (средний показатель без учёта 
самых высоких и низких зарплат) в России на апрель 2023 г. составляла 50 000 руб.
\cite{Salary}, а в Саратове "--- 30 000 руб. \cite{salarySar}. 
Тем не менее, стоит помнить, что без искреннего интереса к профессии достаточно 
непросто приобретать навыки, поэтому зарплата не должна быть ключевым моментом в 
выборе профессии.

\section{ПРОЦЕСС ТЕСТИРОВАНИЯ}
Тестирование программного обеспечения "--- часть процесса разработки, в котором 
проверяется соответствие программы её спецификации и выявляются ошибки \cite{Staroletov}.

Из определения вытекает, что тестирование имеет следующие основные цели:
\begin{enumerate}
    \item Показать разработчику и клиенту, что программное обеспечение отвечает 
    заявленным требованиям.
    \item Найти ситуации, когда программное обеспечение ведет себя ошибочно, 
    нежелательно или не соответствует спецификации.
\end{enumerate}
\subsection{ЭТАПЫ ТЕСТИРОВАНИЯ ПО}
\begin{enumerate}
    \item Планирование и анализ требований. На этой стадии требуется определить: 
    что предстоит тестировать; как много будет работы; какие сложности могут 
    возникнуть и т.п. Чтобы получить ответы на данные вопросы, нужно 
    проанализировать требования. 
    \item Разработка тест"=кейсов. Стадия посвящена разработке, уточнению, 
    доработке и прочим действиям с наборами тест"=кейсов и тестовыми сценариями, 
    которые будут использоваться при непосредственном выполнении тестирования.
    \item Тестирование и фиксация дефектов. Эти действия тесно связаны между 
    собой и фактически выполняются параллельно: дефекты фиксируются сразу по 
    факту их обнаружения в процессе выполнения тест"=кейсов.
    \item Анализ результатов тестирования и отчётность. Данные действия также 
    связаны между собой тесно связаны между собой и выполняются практически 
    параллельно. Формулируемые на стадии анализа результатов выводы напрямую 
    зависят от плана тестирования и стратегии, полученных на предыдущих стадиях 
    \cite{Kulikov}.
\end{enumerate}
Полученные выводы оформляются и служат основой для начальной стадии следующей 
итерации тестирования. Таким образом, цикл замыкается.

На каждом этапе жизненного цикла должны выполняться верификация и валидация 
проекта. Верификация "--- это процесс оценки системы или её компонентов с целью 
определения удовлетворяют ли результаты текущего этапа разработки условиям, 
сформированным в начале этого этапа. Валидация "--- это определение соответствия 
разрабатываемого ПО ожиданиям и потребностям пользователя, требованиям к системе 
\cite{BaseofTest}. Тестирование как инструмент верификации и валидации 
является постоянным процессом и проводится на всех этапах проекта.
\subsection{ОСНОВНЫЕ МЕТОДЫ ТЕСТИРОВАНИЯ}
Согласно устоявшейся терминологии, тестирование с точки зрения использования 
исходного кода программной системы подразделяется на \cite{Ivanova}:
\begin{enumerate}
    \item «Тестирование черного ящика», когда исходный код системы недоступен и 
    производится проверка соответствия ожидаемому поведению;
    \item <<Тестирование белого ящика>>, когда для тестирования доступен ее 
    исходный код;
    \item «Тестирование серого ящика», когда для тестирования доступен ее 
    исходный код частично, например, некоторых компонентов, либо код недоступен, 
    но что"=то известно про внутреннюю структуру (например, используемые 
    алгоритмы) \cite{Import_test}. 
\end{enumerate}
Эти методы не являются взаимозаменяемыми, а дополняют друг друга. Поэтому для 
качественного тестирования продукта рекомендуется использовать тестирование и 
черного, и белого ящика одновременно разными группами тестировщиков.

Если разделять тестирование по уровню тестируемого ПО, то можно выделить 
следующие 4 вида \cite{Bugaenko}:
\begin{enumerate}
    \item Модульное (блочное) тестирование. Нацелено на независимую проверку 
    работы компонент (модулей, блоков, объектов, классов, функций) ПО. 
    Тестирование каждого из модулей выполняется изолированно.
    \item Интеграционное тестирование. Уровень тестирования, направленный на 
    проверку взаимодействия между частями (модулями) приложения. На этапе 
    интеграционного тестирования выполняется поиск ошибок, связанных с 
    трактовкой данных, реализацией интерфейса взаимодействия и совместимостью 
    компонент приложения. Как правило, для интеграционного тестирования 
    применяется метод серого ящика: известны все характеристики взаимосвязей 
    между модулями, но модули закрыты для анализа.
    \item Системное тестирование. На данном уровне завершается проверка 
    реализации приложения. Для системного тестирования применяется подход 
    черного ящика: приложение рассматривается как единое целое, на вход подаются 
    реальные данные, работа приложения анализируется по полученным результатам.
    \item Пользовательское. На данном этапе к тестированию приложения 
    подключаются сторонние участники, включая будущих пользователей и экспертов. 
    По результатам тестирования принимается решение о внедрении проекта \cite{BaseofTest}.
\end{enumerate}
\subsection{ИНСТРУМЕНТЫ, ПРИМЕНЯЕМЫЕ ПРИ ТЕСТИРОВАНИИ}
Исходя из используемых инструментов, тестирование можно разделить на ручное и 
автоматизированное.

Ручное тестирование "--- это метод тестирования, при котором 
инженер"=тестировщик вручную подготавливает тест"=кейсы и выполняет их для 
выявления дефекта в ПО. Это трудоемкая деятельность, которая требует от 
тестировщика обладать определенным набором качеств: быть терпеливым, 
наблюдательным и творческим.

Автоматизированное тестирование "--- это метод тестирования, в котором 
используются средства автоматизации для выполнения набора тестов. Такое 
тестирование программного обеспечения включает в себя разработку тестовых 
сценариев с использованием языков сценариев, таких как Python или JavaScript, 
так что тестовые примеры могут выполняться компьютерами с минимальным 
вмешательством человека \cite{Sharm}. Совместное проектирование и 
разработка тестов могут быть автоматизированы для сокращения человеческих усилий 
и экономии затрат \cite{Krya}. Также с помощью BAT"=файлов для 
Windows и BASH"=скриптов для Linux можно автоматизировать всевозможные задачи: 
от удаления файлов до запуска приложений, что значительно сокращает время по 
сравнению с ручным тестированием. Кроме того, ПО также может сравнивать 
ожидаемые и полученные результаты и генерировать подробные отчёты о тестировании
\cite{Bugaenko}. Цель автоматизации состоит в том, чтобы 
уменьшить количество тестовых примеров, запускаемых вручную, а не полностью 
исключить ручное тестирование \cite{Dustin}.

Важно отметить, что при использовании любого из этих подходов тестировщику 
необходимо одинаково хорошо понимать технологии, применяемые в проекте.
 


\conclusion
В ходе работы была изучена информация о профессии тестировщик.

В настоящее время профессия тестировщика довольно востребована, поскольку 
постоянно появляются новые приложения и программы. Независимо от того, какие 
подходы или методы использует компания, конечная цель всегда одна "--- 
предоставить клиентам наиболее качественный продукт. Продукт тестируют разными 
способами на каждом этапе. Чем раньше найдены ошибки, тем проще и дешевле для 
компании будет их исправить.

Кроме того, профессия тестировщика ПО "--- отличная база для построения 
дальнейшей карьеры в практически любом IT"=направлении. И даже с развитием 
искусственного интеллекта, профессия будет актуальна: искусственный интеллект не 
сможет полностью заменить специалиста по тестированию, поскольку в этой работе 
часто именно человек играет решающую роль.

Тем не менее для собственного развития тестировщик должен быть в тренде 
последних новинок и активно расширять свой кругозор по предметной области.

% Библиографический список, составленный вручную, без использования BibTeX
%
% \begin{thebibliography}{99}
%   \bibitem{Ione} Источник 1.
%   \bibitem{Itwo} Источник 2
% \end{thebibliography}

% Отобразить все источники. Даже те, на которые нет ссылок.
% \nocite{*}

% Меняем inputencoding на лету, чтобы работать с библиографией в кодировке
% `cp1251', в то время как остальной документ находится в кодировке `utf8'
% Credit: Никита Рыданов
\inputencoding{cp1251}
\bibliographystyle{gost780uv}
\bibliography{thesis}
\inputencoding{utf8}

% При использовании biblatex вместо bibtex
% \printbibliography

% Окончание основного документа и начало приложений Каждая последующая секция
% документа будет являться приложением
\appendix

\end{document}
