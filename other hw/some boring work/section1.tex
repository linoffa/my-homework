\subsection{РОЛЬ ТЕСТИРОВЩИКА В РАЗРАБОТКЕ}
Для любой IT-компании, разрабатывающей и поставляющей на рынок свой продукт, 
тестирование является важным и ценным этапом жизненного цикла разработки 
программного обеспечения (ПО). Если процесс тестирования игнорируется то, может 
пострадать как сам продукт, так и компания разработчика в целом, поскольку 
ошибки в ПО могут быть дорогостоящими или даже опасными.

Рассмотрим, как именно тестировщики помогают достичь успеха проекту. Тесное 
сотрудничество тестировщиков с разработчиками, улучшает понимание тестировщиками
каждой части кода и позволяет сконцентрироваться на тех областях, где наиболее 
высок риск возникновения ошибок. В свою очередь, обмен данными полезен и 
разработчикам для наиболее быстрого исправления неполадок. Таким образом, 
взаимодействие между тестировщиками и разработчиками позволяет упростить 
контроль качества и сократить сроки разработки продукта, снизить стоимость 
внесения исправлений \cite{Import_test}.

Кроме того, тестировщики многократно проверяют программное обеспечение перед его 
выпуском. Это помогает в обнаружении ошибок, которые иначе могли бы остаться 
незамеченными.

Поставка ПО хорошего качества, обладающего уникальными функциями, всегда была 
приоритетом в индустрии программного обеспечения. Поэтому участие тестировщика в 
разработке является очень важным.  

\subsection{КВАЛИФИКАЦИЯ И НАВЫКИ ТЕСТИРОВЩИКА}
Несомненно, в профессии важны личные качества человека. Рассмотрим, какие 
являются наиболее важными \cite{SoftSkills}:
\begin{enumerate}
    \item Усидчивость и терпение. Тестировщику необходимо терпеливо искать 
    дефекты в ПО, отслеживая даже самые малейшие отклонения.
    \item Нестандартное мышление. Практически каждый из нас, включая детей и 
    пенсионеров, посещает и использует самые разные сайты компаний ежедневно. 
    Поэтому тестировщик должен уметь представлять самые разнообразные случаи, 
    которые могут возникнуть при взаимодействии клиента с ПО.
    \item Наблюдательность. Необходима, чтобы не пропустить даже самые 
    малейшие отклонения, ведь в дальнейшем это может сильно сказаться на 
    качестве продукта.
    \item Ответственность. Это качество важно тестировщику для того, чтобы он 
    постоянно стремился сделать продукт лучше.
    \item Коммуникабельность и умение работать в команде. Тестировщику для 
    успешной разработки приходится постоянно взаимодействовать с разработчиками, 
    дизайнерами и другими участниками этого процесса.
\end{enumerate}
Кроме того, от тестировщика требуются и навыки такие как:
\begin{enumerate}
    \item Знание основ тестирования, его видов и методов;
    \item Умение составлять тест-кейсы;
    \item Знание языка запросов SQL, умение работать с базами данных;
    \item Знание языков программирования;
    \item Знание систем контроля версий, например, Git.
\end{enumerate}
А также немаловажно владение инструментами ручного и автоматического 
тестирования. Это могут быть:
\begin{enumerate}
    \item Cистемы для создания тест-кейсов и отслеживания ошибок.
    \item Файловые менеджеры, текстовые и XML-редакторы.
    \item Генераторы тестовых данных и другие.
\end{enumerate}
Всё сказанное сочетает в себе хороший тестировщик, помогающий развивать продукт 
в лучшую сторону.
\subsection{ПРОФЕССИЯ «ТЕСТИРОВЩИК» СЕГОДНЯ}
В настоящее время в российских вузах нет отдельных направлений для 
инженеров"=тестировщиков, поэтому начать осваивать данную профессию можно 
самостоятельно или же можно обратиться к набирающему популярность методу 
обучения "--- онлайн"=курсам. Например, «Яндекс Практикум» предлагает пройти 
курс «Инженер по тестированию» и за 4 месяца приобрести навыки, необходимые для 
дальнейшего развития в этой области \cite{Yandex}. 

Условия работы тестировщика имеют довольно большую гибкость. В зависимости от 
конкретной вакансии можно работать в офисе, удалённо, или вовсе на фрилансе.

Немаловажным показателем любой профессии является её уровень заработной платы. 
Так, зарплата специалиста по тестированию зависит преимущественно от его опыта, 
навыков и работодателя, на которого он работает.

Исходя из статистики, медианная заработная плата (средний показатель без учёта 
самых высоких и низких зарплат) в России на апрель 2023 г. составляла 50 000 руб.
\cite{Salary}, а в Саратове "--- 30 000 руб. \cite{salarySar}. 
Тем не менее, стоит помнить, что без искреннего интереса к профессии достаточно 
непросто приобретать навыки, поэтому зарплата не должна быть ключевым моментом в 
выборе профессии.
