\section{ПРОЦЕСС ТЕСТИРОВАНИЯ}
Тестирование программного обеспечения "--- часть процесса разработки, в котором 
проверяется соответствие программы её спецификации и выявляются ошибки \cite{Staroletov}.

Из определения вытекает, что тестирование имеет следующие основные цели:
\begin{enumerate}
    \item Показать разработчику и клиенту, что программное обеспечение отвечает 
    заявленным требованиям.
    \item Найти ситуации, когда программное обеспечение ведет себя ошибочно, 
    нежелательно или не соответствует спецификации.
\end{enumerate}
\subsection{ЭТАПЫ ТЕСТИРОВАНИЯ ПО}
\begin{enumerate}
    \item Планирование и анализ требований. На этой стадии требуется определить: 
    что предстоит тестировать; как много будет работы; какие сложности могут 
    возникнуть и т.п. Чтобы получить ответы на данные вопросы, нужно 
    проанализировать требования. 
    \item Разработка тест"=кейсов. Стадия посвящена разработке, уточнению, 
    доработке и прочим действиям с наборами тест"=кейсов и тестовыми сценариями, 
    которые будут использоваться при непосредственном выполнении тестирования.
    \item Тестирование и фиксация дефектов. Эти действия тесно связаны между 
    собой и фактически выполняются параллельно: дефекты фиксируются сразу по 
    факту их обнаружения в процессе выполнения тест"=кейсов.
    \item Анализ результатов тестирования и отчётность. Данные действия также 
    связаны между собой тесно связаны между собой и выполняются практически 
    параллельно. Формулируемые на стадии анализа результатов выводы напрямую 
    зависят от плана тестирования и стратегии, полученных на предыдущих стадиях 
    \cite{Kulikov}.
\end{enumerate}
Полученные выводы оформляются и служат основой для начальной стадии следующей 
итерации тестирования. Таким образом, цикл замыкается.

На каждом этапе жизненного цикла должны выполняться верификация и валидация 
проекта. Верификация "--- это процесс оценки системы или её компонентов с целью 
определения удовлетворяют ли результаты текущего этапа разработки условиям, 
сформированным в начале этого этапа. Валидация "--- это определение соответствия 
разрабатываемого ПО ожиданиям и потребностям пользователя, требованиям к системе 
\cite{BaseofTest}. Тестирование как инструмент верификации и валидации 
является постоянным процессом и проводится на всех этапах проекта.
\subsection{ОСНОВНЫЕ МЕТОДЫ ТЕСТИРОВАНИЯ}
Согласно устоявшейся терминологии, тестирование с точки зрения использования 
исходного кода программной системы подразделяется на \cite{Ivanova}:
\begin{enumerate}
    \item «Тестирование черного ящика», когда исходный код системы недоступен и 
    производится проверка соответствия ожидаемому поведению;
    \item <<Тестирование белого ящика>>, когда для тестирования доступен ее 
    исходный код;
    \item «Тестирование серого ящика», когда для тестирования доступен ее 
    исходный код частично, например, некоторых компонентов, либо код недоступен, 
    но что"=то известно про внутреннюю структуру (например, используемые 
    алгоритмы) \cite{Import_test}. 
\end{enumerate}
Эти методы не являются взаимозаменяемыми, а дополняют друг друга. Поэтому для 
качественного тестирования продукта рекомендуется использовать тестирование и 
черного, и белого ящика одновременно разными группами тестировщиков.

Если разделять тестирование по уровню тестируемого ПО, то можно выделить 
следующие 4 вида \cite{Bugaenko}:
\begin{enumerate}
    \item Модульное (блочное) тестирование. Нацелено на независимую проверку 
    работы компонент (модулей, блоков, объектов, классов, функций) ПО. 
    Тестирование каждого из модулей выполняется изолированно.
    \item Интеграционное тестирование. Уровень тестирования, направленный на 
    проверку взаимодействия между частями (модулями) приложения. На этапе 
    интеграционного тестирования выполняется поиск ошибок, связанных с 
    трактовкой данных, реализацией интерфейса взаимодействия и совместимостью 
    компонент приложения. Как правило, для интеграционного тестирования 
    применяется метод серого ящика: известны все характеристики взаимосвязей 
    между модулями, но модули закрыты для анализа.
    \item Системное тестирование. На данном уровне завершается проверка 
    реализации приложения. Для системного тестирования применяется подход 
    черного ящика: приложение рассматривается как единое целое, на вход подаются 
    реальные данные, работа приложения анализируется по полученным результатам.
    \item Пользовательское. На данном этапе к тестированию приложения 
    подключаются сторонние участники, включая будущих пользователей и экспертов. 
    По результатам тестирования принимается решение о внедрении проекта \cite{BaseofTest}.
\end{enumerate}
\subsection{ИНСТРУМЕНТЫ, ПРИМЕНЯЕМЫЕ ПРИ ТЕСТИРОВАНИИ}
Исходя из используемых инструментов, тестирование можно разделить на ручное и 
автоматизированное.

Ручное тестирование "--- это метод тестирования, при котором 
инженер"=тестировщик вручную подготавливает тест"=кейсы и выполняет их для 
выявления дефекта в ПО. Это трудоемкая деятельность, которая требует от 
тестировщика обладать определенным набором качеств: быть терпеливым, 
наблюдательным и творческим.

Автоматизированное тестирование "--- это метод тестирования, в котором 
используются средства автоматизации для выполнения набора тестов. Такое 
тестирование программного обеспечения включает в себя разработку тестовых 
сценариев с использованием языков сценариев, таких как Python или JavaScript, 
так что тестовые примеры могут выполняться компьютерами с минимальным 
вмешательством человека \cite{Sharm}. Совместное проектирование и 
разработка тестов могут быть автоматизированы для сокращения человеческих усилий 
и экономии затрат \cite{Krya}. Также с помощью BAT"=файлов для 
Windows и BASH"=скриптов для Linux можно автоматизировать всевозможные задачи: 
от удаления файлов до запуска приложений, что значительно сокращает время по 
сравнению с ручным тестированием. Кроме того, ПО также может сравнивать 
ожидаемые и полученные результаты и генерировать подробные отчёты о тестировании
\cite{Bugaenko}. Цель автоматизации состоит в том, чтобы 
уменьшить количество тестовых примеров, запускаемых вручную, а не полностью 
исключить ручное тестирование \cite{Dustin}.

Важно отметить, что при использовании любого из этих подходов тестировщику 
необходимо одинаково хорошо понимать технологии, применяемые в проекте.
 
