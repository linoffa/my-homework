\documentclass[a4paper, 14pt]{extarticle}
\usepackage[T2A]{fontenc}
\usepackage[english, russian]{babel}
\usepackage[utf8]{inputenc}
\usepackage[left=0.7cm, right=1.2cm, top=1.2cm, bottom=1.2cm, 
    bindingoffset=0cm]{geometry}
\usepackage{fancyhdr}
\usepackage{tempora}
\usepackage{minted}

\author{}
\title{\MakeUppercase{Графы}}
\date{}

\setminted[cpp]{fontsize=\small, breaklines=true, style=friendly}

\begin{document}
\maketitle
\thispagestyle{empty}
\pagestyle{empty}

Функция создания списка смежности графа
\begin{minted}{cpp}
    void create_graph(int n, int m)
    {
        graph.resize(n); // выделение памяти в векторе
        int v, u;
        for (int i = 0; i < m; ++i) // добавление рёбер в список смежности
        { 
            cin >> v >> u;
            graph[v].push_back(u);
            graph[u].push_back(v);
        }

        for (int i = 0; i < n; ++i) 
        {
            sort(graph[i].begin(), graph[i].end()); 
            // сортируем списки смежных вершин для всех вершин
            graph[i].erase(unique(graph[i].begin(), graph[i].end()), graph[i].end()); 
            // удаляем дубликаты, если они есть
        }
    }
\end{minted}

Функция обхода графа в глубину
\begin{minted}{cpp}
    void dfs(int v)
    {
        used[v] = 1;
        cout << v + 1 << ' ';
        for (int u : g[v]) 
            if (!used[u]) 
                dfs(u);
    }
\end{minted}

Функция обхода графа в ширину
\begin{minted}{cpp}
    void bfs(int v)
    {
        queue<int> s;
        s.push(v);
        used[v] = 1;
        cout << v + 1 << ' ';
        while (!s.empty())
        {
            int u = s.front();
            s.pop();
            for (int w : g[u])
            {
                if (used[w]) continue;
                used[w] = 1;
                cout << w + 1 << ' ';
                s.push(w);
            }
        }
    }
\end{minted}
\end{document}