\documentclass[a4paper, 14pt]{extarticle}
\usepackage[T2A]{fontenc}
\usepackage[english, russian]{babel}
\usepackage[utf8]{inputenc}
\usepackage[left=0.7cm, right=1.2cm, top=1.2cm, bottom=1.2cm, 
    bindingoffset=0cm]{geometry}
\usepackage{fancyhdr}
\usepackage{amsmath}
\usepackage{amssymb}
\usepackage{tempora}
\usepackage{minted}

\author{}
\title{\MakeUppercase{Графы}}
\date{}

\setminted[cpp]{fontsize=\small, breaklines=true, style=friendly}

\begin{document}
\maketitle
\thispagestyle{empty}
\pagestyle{empty}

$\checkmark$ Функция создания списка смежности графа
\begin{minted}{cpp}
    void create_graph(int n, int m)
    {
        graph.resize(n); // выделение памяти в векторе
        int v, u;
        for (int i = 0; i < m; ++i) // добавление рёбер в список смежности
        { 
            cin >> v >> u;
            graph[v].push_back(u);
            graph[u].push_back(v);
        }

        for (int i = 0; i < n; ++i) 
        {
            sort(graph[i].begin(), graph[i].end()); 
            // сортируем списки смежных вершин для всех вершин
            graph[i].erase(unique(graph[i].begin(), graph[i].end()), graph[i].end()); 
            // удаляем дубликаты, если они есть
        }
    }
\end{minted}

$\checkmark$ Функция обхода графа в глубину
\begin{minted}{cpp}
    void dfs(int v)
    {
        used[v] = 1;
        cout << v + 1 << ' ';
        for (int u : g[v]) 
            if (!used[u]) 
                dfs(u);
    }
\end{minted}

$\checkmark$ Функция обхода графа в ширину
\begin{minted}{cpp}
    void bfs(int v)
    {
        queue<int> s;
        s.push(v);
        used[v] = 1;
        cout << v + 1 << ' ';
        while (!s.empty())
        {
            int u = s.front();
            s.pop();
            for (int w : g[u])
            {
                if (used[w]) continue;
                used[w] = 1;
                cout << w + 1 << ' ';
                s.push(w);
            }
        }
    }
\end{minted}

$\checkmark$ Функция вставки нового ребра в ориентированный граф
\begin{minted}{cpp}
    void insert(int a, int b)
    { 
        for (auto v : graph[a])
        {
            if (v == b) // если такое ребро уже есть в графе
            { 
                cout << "This edge is in the graph" << endl;
                return;
            }
        }
        graph[a].push_back(b); // добавляем новое ребро
        sort(graph[a].begin(), graph[a].end()); // сортируем имеющиеся значения
    }
\end{minted}

$\checkmark$ Функции для поиска циклов в графе
\begin{minted}{cpp}
    void add_cycles(int start, int end) // функция добавления нового цикла
    { 
        vector<int> temp;
        // восстанавливаем цикл с помощью массива предков
        for (int cur = end; cur != start; cur = pr[cur])
            temp.push_back(cur);
        temp.push_back(start);
        reverse(temp.begin(), temp.end()); // переворачиваем вектор
        cycles.push_back(temp); 
        sort(temp.begin(), temp.end()); // сортируем все вершины цикла
        sorted_cycles.insert(temp); // добавляем в множество
    }

    void find_cycle(int v, int parent = -1) // обход
    { 
        used[v] = 1; // отмечаем вершину посещенной
        pr[v] = parent; // обозначаем вершину предком
        for (int u : graph[v]) 
        {
            if (pr[v] == u) // если текущая вершина является предком, то пропускаем
                continue; 
            if (!used[u]) // если не посещали эту вершину, то продолжаем обход
                find_cycle(u, v);  
            else // добавляем цикл
                add_cycles(u, v); 
        }
        used[v] = 0;
    }
\end{minted}
Задача на поиск кратчайшего пути коня с использованием bfs
\inputminted{cpp}{knights_path.cpp}
\end{document}