\documentclass[a4paper, 14pt]{extarticle}
\usepackage[T2A]{fontenc}
\usepackage[english, russian]{babel}
\usepackage[utf8]{inputenc}
\usepackage[left=0.7cm, right=1.2cm, top=1.2cm, bottom=1.2cm, 
    bindingoffset=0cm]{geometry}
\usepackage{fancyhdr}
\usepackage{amsmath}
\usepackage{amssymb}
\usepackage{tempora}
\usepackage{minted}

\author{}
\title{\MakeUppercase{Hash tables}}
\date{}

\setminted[cpp]{fontsize=\small, breaklines=true, style=perldoc}

\begin{document}
\maketitle
\thispagestyle{empty}
\pagestyle{empty}
\section*{Открытое хеширование}
$\checkmark$ Функция создания хеш-таблицы
\begin{minted}{cpp}
    hashing_table *create(vector<people> x, int size) // создание хеш-таблицы
    {
        hashing_table *table = new hashing_table[size]; // создаем новый элемент
        int num = 0;
        for (int i = 0; i < 25; i++)
        {
            num = hash_function(x[i].experience, size); 
            push(table[num].h, table[num].t, x[i]); // добавляем в ячейку в зависимости от значения хеш-функции
        }
        return table;
    }
\end{minted}
$\checkmark$ Функция удаления эл-та таблицы
\begin{minted}{cpp}
    void del(int m, hashing_table *&table, int size) 
    { 
        int num = hash_function(m, size); // вычисляем значение хеш-функции
        list *p = table[num].h;
        while(p)
        {
            if (p->inf.experience == m) // если нашли эл-т 
            {
                del_node(table[num].h, table[num].t, p); // удаляем
                return;
            }
            p = p->next; // иначе ищем дальше
        }
    }
\end{minted}
$\checkmark$ Функция поиска эл-та таблицы
\begin{minted}{cpp}
    list *find(int m, hashing_table *&table, int size)
    {
        int num = hash_function(m, size); // вычисляем значение хеш-функции
        list *p = table[num].h;
        while(p)
        {
            if (p->inf.experience == m) // если нашли эл-т 
                return p; // возвращаем указатель
            p = p->next; // идём дальше
        }
        return NULL;
    }
\end{minted}
\section*{Закрытое хеширование}
$\checkmark$ Функция создания хеш-таблицы
\begin{minted}{cpp}
    hashing_table *create(vector<people> x, int size)
    {
        hashing_table *table = new hashing_table[size]; // создаем новый элемент
        int num = 0;
        for (int i = 0; i < 25; i++)
        {
            num = hash_function(x[i].date_of_birth.yy, size); 
            while (num < size)
            {
                if (!table[num].h) // если ячейка занята, то переходим к следующей
                {
                    push(table[num].h, table[num].t, x[i]);
                    break;
                }
                else
                    num++;
            }
        }
        return table;
    }
\end{minted}
$\checkmark$ Функция поиска эл-та таблицы
\begin{minted}{cpp}
    void find(int m, hashing_table *&table, int size)
    {
        int num = hash_function(m, size); // вычисляем значение хеш-функции
        list *p = table[num].h;
        while(p)
        {
            list *p = table[num].h;
            if (p->inf.date_of_birth.yy == m) // если нашли эл-т 
                print_about_people(p->inf); // выводим нужную информацию
            p = p->next; // идём дальше
            num++;
        }
    }
\end{minted}
\end{document}