\documentclass[a4paper, 14 pt]{extarticle}
\usepackage[T2A]{fontenc}
\usepackage[english, russian]{babel}
\usepackage[utf8]{inputenc}
\usepackage[left=0.7cm, right=1.2cm, top=1.2cm, bottom=1.2cm, 
    bindingoffset=0cm]{geometry}
\usepackage{fancyhdr}
\usepackage{tempora}
    
\title{\textbf{Глоссарий}}
\author{Ангелина Федина, 151 гр.}
\date{}

\begin{document}

\maketitle
\thispagestyle{empty}
\pagestyle{empty}

\begin{enumerate}
    \item[] \large{\textbf{A}}
    \item \textbf{AlphaGo} (\textit{Алексей Кузьмин}) "--- программа для игры в 
    го, разработанная компанией Google DeepMind в 2015 году. AlphaGo стала 
    первой в мире программой, которая выиграла матч у профессионального игрока
    в го.
    \item \textbf{AlphaZero} (\textit{Алексей Кузьмин}) "--- нейронная сеть, 
    разработанная компанией DeepMind, которая использует обобщенный подход 
    AlphaGo Zero.
    \item \textbf{API, Application Programming Interface} 
    (\textit{Никита Барабанов}) "--- совокупность инструментов и функций в 
    виде интерфейса для создания новых приложений, благодаря которому одна 
    программа будет взаимодействовать с другой. 
    \item[] \textbf{B}
    \item \textbf{Backend} (\textit{Никита Барабанов}) "--- внутренняя часть 
    продукта, которая находится на сервере и скрыта от пользователей.
    \item \textbf{Backlog} (\textit{Никита Барабанов}) "--- список задач или 
    требований к продукту, которые нужно реализовать.
    \item \textbf{Big Data} (\textit{Алексей Кузьмин}) "--- 
    структурированные или неструктурированные массивы данных большого объема.
    \item \textbf{Business intelligence} (\textit{Алексей Кузьмин}) "--- поиск 
    оптимальных бизнес"=решений с помощью обработки большого объема 
    неструктурированных данных для полноты картины бизнеса и принятия правильных
    как операционных, так и стратегических решений.
    \item[] \textbf{С}
    \item \textbf{ChatGPT} (\textit{Никита Рыданов}) "--- чат"=бот, основанный 
    на новой версии нейросетевой модели GPT"=3.5. Его разработала компания 
    OpenAI в сотрудничестве с Microsoft.
    \item[] \textbf{D}
    \item \textbf{Data mining} (\textit{Алексей Кузьмин}) "--- интеллектуальный 
    анализ данных с целью выявления закономерностей, технологии обнаружения в 
    сырых данных ранее неизвестных и практически полезных данных.
    \item \textbf{Data Science} (\textit{Алексей Кузьмин}) "--- 
    междисциплинарная область на стыке статистики, математики, системного 
    анализа и машинного обучения, которая охватывает все этапы работы с данными.
    \item \textbf{Deep Learning} (\textit{Алексей Кузьмин}) "--- совокупность 
    широкого семейства методов машинного обучения, основанных на имитации работы
    человеческого мозга в процессе обработки данных и создания паттернов, 
    используемых для принятия решений. 
    \item \textbf{Deploy} (\textit{Никита Барабанов}) "--- размещение 
    готовой версии программного обеспечения на платформе, доступной для 
    пользователей.
    \item \textbf{DevOps} (\textit{Никита Барабанов}) "---  методология 
    автоматизации технологических процессов сборки, настройки и развёртывания 
    программного обеспечения. Методология предполагает активное взаимодействие 
    специалистов по разработке со специалистами по 
    информационно"=технологическому обслуживанию и взаимную интеграцию их 
    технологических процессов друг в друга для обеспечения высокого качества
    программного продукта.
    \item[] \textbf{E}
    \item \textbf{ERP} (\textit{Ростислав, Сибинтек}) "--- программное 
    обеспечение, помогающее предприятиям автоматизировать основные 
    бизнес"=процессы и управлять ими для достижения оптимальной 
    производительности. \\
    \\
    \item[] \textbf{F}
    \item \textbf{Framework} (\textit{Алексей Кузьмин}) "--- программная 
    платформа, определяющая структуру программной системы; программное 
    обеспечение, облегчающее разработку и объединение разных компонентов 
    большого программного проекта.
    \item \textbf{Frontend} (\textit{Никита Барабанов}) "--- клиентская часть 
    продукта, интерфейс, с которым взаимодействует пользователь.
    \item[] \textbf{G}
    \item \textbf{GAN, Generative Adversarial Nets} (\textit{Алексей Кузьмин}) 
    "--- алгоритм машинного обучения, построенный на комбинации из двух 
    нейронных сетей: генеративная модель, которая строит приближение 
    распределения данных, и дискриминативная модель, оценивающая вероятность, 
    что образец пришел из тренировочных данных, а не сгенерированных 
    генеративной моделью G. Впервые такие сети были представлены Иэном Гудфеллоу
    в 2014 году.
    \item \textbf{GPT"=model} (\textit{Михаил Чернигин}) "--- тип нейронных
    языковых моделей, впервые представленных компанией OpenAI, которые обучаются
    на больших наборах текстовых данных, чтобы генерировать текст, схожий с 
    человеческим.
    \item[] \textbf{M}
    \item \textbf{MRP} (\textit{Ростислав, Сибинтек}) "--- cистема планирования 
    потребностей в материалах, одна из наиболее популярных в мире логистических 
    концепций, на основе которой разработано и функционирует большое число 
    микрологистических систем.
    \item \textbf{MySQL} (\textit{Иван Жадаев}) "--- система управления 
    реляционными базами данных с открытым исходным кодом, основанная на языке 
    структурированных запросов (SQL) и являющаяся основным решением для хранения 
    реляционных данных для веб"=сайтов и приложений. \\
    \\
    \\
    \item[] \textbf{O}
    \item \textbf{Open Source} (\textit{Никита Рыданов}) "--- программное 
    обеспечение, исходный код которого доступен для всех пользователей.
    \item[] \textbf{P}
    \item \textbf{Pipeline} (\textit{Никита Барабанов}) "--- документ, 
    визуализирующий процесс разработки продукта. 
    \item \textbf{Prompt} (\textit{Никита Рыданов}) "--- текстовый запрос для 
    нейросети, на основе которого генерируются текстовые или графические данные.
    \item \textbf{Proxy} (\textit{Игорь Юрин}) "--- промежуточный сервер между 
    пользователем интернета и серверами, откуда запрашивается информация. 
    \item[] \textbf{R}
    \item \textbf{Roadmap} (\textit{Павел Пасеков}) "--- последовательность 
    важных задач, которые нужно сделать за определённый период, чтобы выпустить 
    продукт на рынок. Чаще всего этот инструмент используют в сфере IT и 
    маркетинга. 
    \item[] \textbf{S}
    \item \textbf{SOA, Service"=oriented architecture} 
    (\textit{Никита Барабанов}) "---  метод разработки программного обеспечения,
    который использует программные компоненты, называемые сервисами, для 
    создания бизнес"=приложений. Каждый сервис предоставляет бизнес-возможности,
    и сервисы также могут взаимодействовать друг с другом на разных платформах и
    языках.
    \item[] \textbf{T}
    \item \textbf{TCP/IP} (\textit{Иван Жадаев}) "--- набор протоколов, который 
    задает стандарты связи между компьютерами и содержит подробные соглашения о 
    маршрутизации и межсетевом взаимодействии.
    \item \textbf{Team Lead} (\textit{Алексей Кузьмин}) "--- специалист 
    координирующий деятельность команды разработчиков, распределяет сферы 
    ответственности, взаимодействует с заказчиком, планирует и организует 
    обучение специалистов.
    \item[] \textbf{W}
    \item \textbf{Workflow} (\textit{Ростислав, Сибинтек}) "--- полная или 
    частичная автоматизация бизнес"=процессов в организации, при которой 
    документы, информация, задачи или задания передаются от одного участника 
    бизнес"=процесса к другому для выполнения действий согласно набору 
    руководящих правил в предусмотренное тайминг"=планом время. 
    \item[] \textbf{А}
    \item \textbf{Аппроксимация} (\textit{Алексей Кузьмин}) "--- это метод 
    вычислений, используемый в математике, заключающийся в том, что сложные 
    математические объекты при расчетах заменяются более простыми. 
    \item[] \textbf{Б} 
    \item \textbf{Баг} (\textit{Никита Барабанов}) "--- ошибка в программе или 
    системе, из"=за которой программа выдает неожиданное поведение и, как 
    следствие, результат.   
    \item[] \textbf{В} 
    \item \textbf{Веб"=сервер} (\textit{Игорь Юрин}) "--- сервер, отвечающий за 
    приём и обработку запросов (HTTP/HTTPS"=запросов) от клиентов к веб"=сайту. 
    В качестве клиентов обычно выступают различные веб"=браузеры.
    \item \textbf{Верификация} (\textit{Никита, КРЭТ КБПА}) "--- процесс оценки 
    системы или её компонентов с целью определения удовлетворяют ли результаты 
    текущего этапа разработки условиям, сформированным в начале этого этапа.
    \item[] \textbf{Д} 
    \item \textbf{Декомпозировать} (\textit{Никита Барабанов}) "--- разбивать 
    одну большую задачу на меньшие, связанные между собой.
    \item[] \textbf{И} 
    \item \textbf{Интернет вещей} (\textit{Игорь Юрин}) "--- сеть физических 
    устройств, которые подключены к другим устройствам и службам через Интернет 
    или другую сеть и обмениваются с ними данными.
    \item \textbf{Интеграционное тестирование} (\textit{Никита, КРЭТ КБПА}) "---
    Уровень тестирования, направленный на проверку взаимодействия между частями 
    (модулями) приложения. 
    \item[] \textbf{К}
    \item \textbf{Кластеризация} (\textit{Алексей Кузьмин}) "--- 
    задача неконтролируемого машинного обучения, которая группирует отдельные 
    экземпляры данных в кластеры со сходными характеристиками.
    \item \textbf{Компьютерная безопасность} (\textit{Игорь Юрин}) "--- меры 
    безопасности, применяемые для защиты вычислительных устройств (компьютеры, 
    смартфоны и другие), а также компьютерных сетей (частных и публичных сетей, 
    включая Интернет). 
    \item[] \textbf{Л}
    \item \textbf{Лог} (\textit{Никита Барабанов}) "--- текстовый файл с 
    информацией о действиях программного обеспечения или пользователей, который 
    хранится на компьютере или сервере. Это хронология событий и их источников, 
    ошибок и причин, по которым они произошли.
    \item[] \textbf{М}
    \item \textbf{Майнинг} (\textit{Игорь Юрин}) "--- деятельность по созданию 
    новых блоков в блокчейне для обеспечения функционирования криптовалютных 
    платформ. За создание очередной структурной единицы обычно предусмотрено 
    вознаграждение за счёт новых (эмитированных) единиц криптовалюты и/или 
    комиссионных сборов.
    \item \textbf{Метапрограммирование} (\textit{Никита, КРЭТ КБПА}) "--- 
    процесс создания метапрограмм для формирования или модификации текста 
    целевой программы. Метапрограммы "--- программы, в результате работы которых 
    формируется необходимый исходный текст целевой программы.
    \item \textbf{Микросервис} (\textit{Никита Барабанов}) "--- это 
    архитектурный и организационный подход к разработке программного 
    обеспечения, при котором программное обеспечение состоит из небольших 
    независимых сервисов, взаимодействующих через четко определенные API.
    \item \textbf{Монолит} (\textit{Никита Барабанов}) "--- это традиционная 
    модель программного обеспечения, которая представляет собой единый модуль, 
    работающий автономно и независимо от других приложений.
    \item[] \textbf{П}
    \item \textbf{Паттерн проектирования} (\textit{Никита Рыданов}) "--- 
    повторяемая архитектурная конструкция в сфере проектирования программного 
    обеспечения, предлагающая решение проблемы проектирования в рамках 
    некоторого часто возникающего контекста.
    \item \textbf{Персептрон} (\textit{Алексей Кузьмин}) "--- математическая или
    компьютерная модель восприятия информации мозгом (кибернетическая модель 
    мозга), предложенная Фрэнком Розенблаттом в 1958 году и впервые 
    реализованная в виде электронной машины «Марк"=1» в 1960 году. Перцептрон 
    стал одной из первых моделей нейросетей, а «Марк"=1» "--- первым в мире 
    нейрокомпьютером.
    \item[] \textbf{Р}
    \item \textbf{Рекуррентная нейронная сеть} (\textit{Алексей Кузьмин}) "--- 
    вид нейронных сетей, где связи между элементами образуют направленную 
    последовательность.
    \item \textbf{Рефракторинг кода} (\textit{Никита Барабанов}) "--- это 
    процесс изменения кода, с целью упростить его обслуживание, понимание и 
    расширение, при этом не изменяя его поведение. 
    \item[] \textbf{С}
    \item \textbf{Свёрточная нейронная сеть} (\textit{Алексей Кузьмин}) "--- 
    специальная архитектура искусственных нейронных сетей, предложенная Яном 
    Лекуном в 1988 году и нацеленная на эффективное распознавание образов. 
    \item \textbf{Сокет} (\textit{Леонид Сорокин}) "--- название программного 
    интерфейса для обеспечения обмена данными между процессами. Процессы при 
    таком обмене могут исполняться как на одной ЭВМ, так и на различных ЭВМ, 
    связанных между собой только сетью. 
    \item \textbf{СУБД, Система управления базами данных} 
    (\textit{Ростислав, Сибинтек}) "--- набор программ, которые управляют 
    структурой базыданных и контролируют доступ к данным, хранящимся в ней. 
    \item[] \textbf{Т}
    \item \textbf{Толстый веб"=клиент} (\textit{Иван Жадаев}) "--- приложение, 
    обеспечивающее расширенную функциональность независимо от центрального 
    сервера. Часто сервер в этом случае является лишь хранилищем данных, а вся 
    работа по обработке и представлению этих данных переносится на машину 
    клиента.
    \item \textbf{Тонкий веб"=клиент} (\textit{Иван Жадаев}) "--- 
    программа"=клиент в сетях с клиент"=серверной или терминальной архитектурой, 
    которая переносит все или большую часть задач по обработке информации на 
    сервер.
    \item \textbf{Трансформер} (\textit{Никита Рыданов}) "--- один из типов 
    архитектуры глубоких нейронных сетей, включающмй в себя кодирующий и 
    декодирующий элементы. Используется преимущественно в областях обработки 
    естественного языка (NLP) и компьютерного зрения (CV).
    \item \textbf{Троян} (\textit{Игорь Юрин}) "--- это вредоносное программное 
    обеспечение, которое маскирует свое истинное назначение. При этом, в отличие 
    от вируса, троян не способен самостоятельно дублировать или заражать файлы. 
    \item[] \textbf{Ч}
    \item \textbf{Чёрный ящик} (\textit{Максим, КРЭТ КБПА}) "--- метод 
    тестирования, при котором исходный код системы недоступен и производится 
    проверка соответствия ожидаемому поведению.
    \item[] \textbf{Э}
    \item \textbf{Эпоха обучения} (\textit{Алексей Кузьмин}) "--- одна итерация 
    в процессе обучения нейронной сети, включающая предъявление всех примеров из 
    обучающего множества и, возможно, проверку качества обучения на контрольном 
    множестве.\\
    \item[] \textbf{Я}
    \item \textbf{Язык запросов} (\textit{Иван Жадаев}) "--- искусственный язык, 
    на котором делаются запросы к базам данных и информационно"=поисковым 
    системам.

\end{enumerate}

\end{document}