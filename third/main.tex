\documentclass[coursework]{SCWorks}
% Тип обучения (одно из значений):
%    bachelor   - бакалавриат (по умолчанию)
%    spec       - специальность
%    master     - магистратура
% Форма обучения (одно из значений):
%    och        - очное (по умолчанию)
%    zaoch      - заочное
% Тип работы (одно из значений):
%    coursework - курсовая работа (по умолчанию)
%    referat    - реферат
%  * otchet     - универсальный отчет
%  * nirjournal - журнал НИР
%  * digital    - итоговая работа для цифровой кафдры
%    diploma    - дипломная работа
%    pract      - отчет о научно-исследовательской работе
%    autoref    - автореферат выпускной работы
%    assignment - задание на выпускную квалификационную работу
%    review     - отзыв руководителя
%    critique   - рецензия на выпускную работу
% Включение шрифта
%    times      - включение шрифта Times New Roman (если установлен)
%                 по умолчанию выключен
\usepackage{preamble}
\graphicspath{{./requirements/pics}}

\setminted[cpp]{fontsize=\small, breaklines=true, style=bw, linenos}
\setminted[python]{fontsize=\small, breaklines=true, style=bw, linenos}
\setcounter{section}{4}
\setcounter{subsection}{3}

\begin{document}

% Кафедра (в родительном падеже)
\chair{математической кибернетики и компьютерных наук}

% Тема работы
\title{ЛЕКСИЧЕСКИЙ И СИНТАКСИЧЕСКИЙ АНАЛИЗ ВЫРАЖЕНИЙ}

% Курс
\course{2}

% Группа
\group{251}

% Факультет (в родительном падеже) (по умолчанию "факультета КНиИТ")
% \department{факультета КНиИТ}

% Специальность/направление код - наименование
% \napravlenie{02.03.02 "--- Фундаментальная информатика и информационные технологии}
% \napravlenie{02.03.01 "--- Математическое обеспечение и администрирование информационных систем}
% \napravlenie{09.03.01 "--- Информатика и вычислительная техника}
\napravlenie{09.03.04 "--- Программная инженерия}
% \napravlenie{10.05.01 "--- Компьютерная безопасность}

% Для студентки. Для работы студента следующая команда не нужна.
% \studenttitle{Студентки}

% Фамилия, имя, отчество в родительном падеже
\author{Рыданова Никиты Сергеевича}

% Заведующий кафедрой 
\chtitle{доцент, к.\,ф.-м.\,н.}
\chname{С.\,В.\,Миронов}

% Руководитель ДПП ПП для цифровой кафедры (перекрывает заведующего кафедры)
% \chpretitle{
%     заведующий кафедрой математических основ информатики и олимпиадного\\
%     программирования на базе МАОУ <<Ф"=Т лицей №1>>
% }
% \chtitle{г. Саратов, к.\,ф.-м.\,н., доцент}
% \chname{Кондратова\, Ю.\,Н.}

% Научный руководитель (для реферата преподаватель проверяющий работу)
\satitle{доцент, к.\,ф.-м.\,н.} %должность, степень, звание
\saname{Г.\,Г.\,Наркайтис}

% Руководитель практики от организации (руководитель для цифровой кафедры)
\patitle{доцент, к.\,ф.-м.\,н.}
\paname{С.\,В.\,Миронов}

% Руководитель НИР
\nirtitle{доцент, к.\,п.\,н.} % степень, звание
\nirname{В.\,А.\,Векслер}

% Семестр (только для практики, для остальных типов работ не используется)
\term{2}

% Наименование практики (только для практики, для остальных типов работ не
% используется)
\practtype{учебная}

% Продолжительность практики (количество недель) (только для практики, для
% остальных типов работ не используется)
\duration{2}

% Даты начала и окончания практики (только для практики, для остальных типов
% работ не используется)
\practStart{01.07.2022}
\practFinish{13.01.2023}

% Год выполнения отчета
\date{2021}

\maketitle

% Включение нумерации рисунков, формул и таблиц по разделам (по умолчанию -
% нумерация сквозная) (допускается оба вида нумерации)
\secNumbering

%\tableofcontents

% Раздел "Обозначения и сокращения". Может отсутствовать в работе
% \abbreviations
% \begin{description}
%     \item ... "--- ...
%     \item ... "--- ...
% \end{description}

% Раздел "Определения". Может отсутствовать в работе
% \definitions

% Раздел "Определения, обозначения и сокращения". Может отсутствовать в работе.
% Если присутствует, то заменяет собой разделы "Обозначения и сокращения" и
% "Определения"
% \defabbr

%\intro

% После введения — серии \section, \subsection и т.д.
\subsection{Управление памятью на основе регионов}
\subsubsection{Мотивировка}
Текущая реализация абстрактного синтаксического дерева имеет следующие 
недостатки:
\begin{enumerate}
    \item Выделение памяти стандартным методом может значительно фрагментировать
    оперативную память, затрудняя доступ к ней.
    \item Любое выделение и удаление памяти требует вмешательства системных
    вызовов, что может стать причиной дополнительных издержек во время
    работы программы.
    \item Программист не имеет возможности ручного управления выделяемой им
    памятью.
\end{enumerate}

Избавиться от этих недостатков можно используя различные оптимизации. В рамках 
этой работы воспользуемся управлением памятью на основе, так называемых, 
регионов (арен, зон) \cite{JohnLevine}.

Под регионом далее будем понимать непрерывную область памяти, содержащую внутри 
себя объекты. При запуске программы выделим регион некоторого размера, при 
необходимости увеличивая его размер в некоторое постоянное
число раз.

Этот подход имеет следующие преимущества:
\begin{enumerate}
    \item Элементы располагаются последовательно, в связи с чем минимизируется
    фрагментация и упрощается доступ к объектам.
    \item Выделение и освобождение памяти выполняется с минимальными издержками
    \item Программисту предоставляется большая свобода для управления выделенной
    памятью.
\end{enumerate}

\subsubsection{Построение}
Формально определим требования к системе:
\begin{enumerate}
    \item Регион должен представлять из себя некоторый непрерывный участок 
    размера $n$ байт (в начальный момент времени размер равен некоторой 
    начальной величине $n_0$).
    \item При обращении к региону он должен предоставить $k$ байт памяти и 
    вернуть некоторый идентификатор этого участка для последующего обращения.
    \item При заполнении региона должна быть возможность увеличить объем 
    доступной памяти в некоторое число раз, которое далее будем называть
    коэффициентом увеличения.
    \item Должна быть доступна возможность эффективного освобождения всей 
    выделенной регионом памяти.
\end{enumerate}

Единственной сложной операцией над регионом является его увеличение.
Так как выделение нового участка потенциально может сопровождаться изменением 
адресов объектов, то необходимо организовать доступ к ним независимо от 
первоначального адреса. Для этого для каждого объекта будем получать доступ к 
нему через некоторый индекс. 

Кроме того, коэффициент увеличения должен быть выбран таким образом, чтобы был 
соблюден баланс между оптимальным объемом выделенной памяти и частотой системных
вызовов.

\subsubsection{Определение структуры}
Определим нашу структуру следующим образом:
\begin{minted}{cpp}
    typedef struct arena {
        // Указатель на начало региона
        struct node* arena;
        // Размер региона
        unsigned int size;
        // Объем выделенной регионом памяти
        unsigned int allocated;
    } arena;
\end{minted}

\subsubsection{Инициализация}
Теперь определим функцию \verb|arena_construct|, выполняющую начальную 
инициализацию состояния региона:
\begin{minted}{cpp}
    int arena_construct (arena* arena) {
        // Начальный размер региона равен некоторой постоянной, равной, DEFAULT_ARENA_SIZE
        arena->size = DEFAULT_ARENA_SIZE;
        arena->allocated = 0;
        // Выделим необходимое число памяти
        arena->arena = malloc(sizeof(node) * DEFAULT_ARENA_SIZE);
        // Если выделение прошло неудачно - вернем в качестве кода ошибки отличное, от 0 значение.
        if (arena->arena == NULL) {
            return (!0);
        }
    return 0;
    }
\end{minted}

\subsubsection{Выделение памяти}
После выделения некоторого объема памяти возможно обращение к ней.
Определим это обращение с помощью функции \verb|arena_allocate|:
\begin{minted}{cpp}
    int arena_allocate (arena* arena, unsigned int count) {
        // Если места в регионе недостаточно
        if (arena->allocated + count >= arena->size) {
            // Определим новый размер региона
            unsigned int newSize = MULTIPLY_FACTOR * arena->size;
            // Выделим регион большего размера и освободим ранее занятую память
            node* newArena = realloc(arena->arena,
            newSize * sizeof(node));
            if (NULL == newArena) {
                return -1;
            }
            arena->arena = newArena;
            arena->size = newSize;
        }
        // В качестве результата вернем индекс первого свободного участка региона
        unsigned int result = arena->allocated;
        // Сместим индекс на объем выделенной памяти
        arena->allocated += count;
        // Вернем результат
        return result;
    }
\end{minted}
Отметим, что наиболее часто значением \verb|MULTIPLY_FACTOR| оказывается числа
1.5 и 2. Это позволяет достичь амортизационно константного времени выполнения 
операции выделения памяти \cite{matrix_87}.

\subsubsection{Освобождение выделенной памяти}
Наконец, реализуем освобождение выделенной региону памяти с помощью функции 
\verb|arena_free|
\begin{minted}{cpp}
    void arena_free (arena* arena) {
        if (arena->arena != NULL)
            free(arena->arena);
        arena->arena = NULL;
    }
\end{minted}

\subsubsection{Модификация абстрактного синтаксического дерева}
Осталось изменить исходный код программы, чтобы обеспечить выделение памяти с 
помощью полученной нами структуры данных.

Для этого воспользуемся директивой \texttt{\%param} и заявим в качестве 
параметра переменную типа \texttt{arena*}. В функциях \texttt{eval}, 
texttt{newnum}, \texttt{newast} внесем изменения, чтобы обеспечить выделение 
памятью с помощью написанных ранее функций.

С полным кодом программы можно ознакомиться в приложении А.

\subsubsection{Сборка проекта}
Теперь проект можно собрать, незначительно изменив \texttt{Makefile}:
\begin{minted}{cpp}
    calc.out: calc.l calc.y arena_ast.h
        bison -d calc.y
        flex calc.l
        cc -o $@ calc.tab.c lex.yy.c arena_ast.c arena.c
\end{minted}
и запустить. Результат работы программы представлен на рис. \ref{img:demon}
\begin{figure}[H]
    \centering
    \includegraphics[width=0.65\textwidth]{naivetest.png}
    \caption{Демонстрация работы программы}
    \label{img:demon}
\end{figure}    

\section{Сравнение полученных реализаций}
Проведем анализ производительности полученных версий анализатора. В качестве 
данных для тестирования возьмем выражения вида 
$\underbrace{2 + 2 + 2 + \cdots + 2}_{n}$ для $n = 1 \dots 100$ с шагом 1. Для 
вычисления времени выполнения воспользуемся библиотекой \texttt{time} 
Python 3.9.5. Автоматизацию обеспечим с помощью библиотеки \texttt{subprocess}. 
Получим следующий код:
\inputminted{python}{requirements/src/test.py}

Кроме того, отметим, что в ранее написанные программы были внесены некоторые 
изменения для проведения эксперимента. Ознакомиться с ними можно в приложении А.

Ознакомиться с полным исходным кодом программы, осуществляющей исследование 
производительности можно в приложении Б.

Для большей наглядности графики интерполированы полиномом с помощью функции 
\texttt{polyfit} библиотеки \texttt{numpy}.

Ознакомиться с полным исходным кодом программы, осуществляющей анализ полученных 
результатов можно в приложении В.

Результаты исследования изображены на рис. \ref{img:benchmark}:
\begin{figure}[H]
    \centering
    \includegraphics[width=0.65\textwidth]{benchmark.png}
    \caption{Сравнение полученных результатов}
    \label{img:benchmark}
\end{figure} 
Исследование показало, что использование абстрактных синтаксических деревьев 
позволяет уменьшить время работы программы более чем в 5 раз, что существенно 
заметно для выражений любой длины.

Также из графиков видно, что в рамках данной работы не удалось добиться большей 
производительности при управлении памятью на основе регионов. Тем не менее, она 
все еще может считаться более предпочительной ввиду перечисленных ранее 
преимуществ.

\conclusion

В ходе данной работы:
\begin{enumerate}
    \item Были изучены теоретические основы построения лексических и 
    синтаксических анализаторов.
    \item Проанализированы особенности реализации лексических и синтаксических 
    анализаторов.
    \item Были изучены принципы работы генераторов лексического и 
    синтаксического анализа на примере Flex и GNU Bison. 
    \item Были созданы лексический и синтаксический анализаторы для анализа
    математического выражения.
    \item Было изучено понятие абстрактного синтаксического дерева.
    \item Проведен анализ производительности полученных реализаций.    
\end{enumerate}
Таким образом, все поставленные в рамках работы задачи выполнены
Результаты исследования показали, что абстрактные синтаксические деревья 
позволяют добиться увеличения производительности в 5–6 раз.

А это, в свою очередь, позволяет утверждать о том, что концепция абстрактных 
синтаксических деревьев является крайне важной в информатике и ее приложениях, 
в частности, при создании синтаксических анализаторов.

% Библиографический список, составленный вручную, без использования BibTeX
%
% \begin{thebibliography}{99}
%   \bibitem{Ione} Источник 1.
%   \bibitem{Itwo} Источник 2
% \end{thebibliography}

% Отобразить все источники. Даже те, на которые нет ссылок.
% \nocite{*}

% Меняем inputencoding на лету, чтобы работать с библиографией в кодировке
% `cp1251', в то время как остальной документ находится в кодировке `utf8'
% Credit: Никита Рыданов
\inputencoding{cp1251}
\bibliographystyle{gost780uv}
\bibliography{thesis}
\inputencoding{utf8}

% При использовании biblatex вместо bibtex
% \printbibliography

% Окончание основного документа и начало приложений Каждая последующая секция
% документа будет являться приложением
\appendix

\end{document}
