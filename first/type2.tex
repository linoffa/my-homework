Лермонтов любил собак. Еще он любил Наталью Николаевну Пушкину. Только больше 
всего он любил самого Пушкина. Читал его стихи и всегда плакал. Поплачет, а 
потом вытащит саблю и давай рубить подушки. Тут и любимая собачка не попадайся 
под руку "--- штук десять так"=то зарубил. А Пушкин ни от каких не плакал. Ни
за что.

Однажды Лермонтов купил яблок, пришел на Тверской бульвар и стал угощать 
присутствующих дам. Все брали и говорили <<мерси>>. Когда же подошла Наталья 
Николаевна с сестрой Александриной, от волненья он так задрожал, что яблоко 
упало к ее ногам (Натальи Николаевны, а не Александрины). Одна из собак схватила
яблоко и бросилась бежать. Александрина, конечно, побежала за ней. Они были одни
"--- впервые в жизни (Лермонтов, конечно, а не Александрина с собачкой). Кстати,
она (Александрина) ее не догнала.

Снится однажды Герцену сон. Будто иммигрировал он в Лондон и живется ему там 
очень хорошо. Купил он, будто, собаку бульдожей породы. И до того злющий
пес "--- сил нет. Кого увидит, на того бросается. И уж если догонит, вцепится 
мертвой хваткой, все, можешь бежать заказывать панихиду. И вдруг, будто он уже
не в Лондоне, а в Москве. Идет по Тверскому бульвару, чудище свое на поводке 
держит, а навстречу Лев Толстой. И надо же, тут на самом интересном месте пришли
декабристы и разбудили.

Гоголь только под конец жизни о душе задумался, а смолоду у него вовсе совести 
не было. Однажды невесту в карты проиграл и не отдал.

Однажды Гоголь переоделся Пушкиным и пришел в гости к Майкову. Майков усадил его
в кресло и угощает пустым чаем. <<Поверите ли, "--- говорит, "--- Александр 
Сергеевич, куска сахару в доме нет. Давеча Гоголь приходил и все съел>>. Гоголь 
ему ничего не сказал.

Однажды Гоголь переоделся Пушкиным, напялил сверху львиную шкуру и поехал в 
маскарад. Федор Михайлович Достоевский, царствие ему небесное, увидел его и 
кричит: <<Спорим, Лев Толстой! Спорим, Лев Толстой!>>.

Однажды Гоголь переоделся Пушкиным и пришел в гости к Вяземскому. Выглянул в 
окно и видит: Толстой Герцена костылем лупит, а кругом детишки стоят, смеются. 
Он пожалел Герцена и заплакал. Тогда Вяземский понял, что перед ним не Пушкин.

Гоголь читал драму Пушкина <<Борис Годунов>> и приговаривал: <<Ай да Пушкин, 
действительно, сукин сын>>.

Однажды Чернышевский увидел из окна своей мансарды, как Лермонтов вскочил на 
коня и крикнул: <<В пассаж!>>. <<Ну и что же? "--- подумал Чернышевский, "--- 
вот, бог даст, революция будет, тогда и я так крикну>>. И стал репетировать 
перед зеркалом, повторяя на разные манеры: <<В пассаж. В пассажж. В пассажжж. В 
па"=а­ссажжж. В ПАССА"=А"=А"=А"=АЖЖЖ!!!>>.