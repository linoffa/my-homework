У Вяземского была квартира окнами на Тверской бульвар. Пушкин очень любил ходить
к нему в гости. Придет "--- и сразу прыг на подоконник, свесится из окна и 
смотрит. Чай ему тоже туда, на окно, подавали. Иной раз там и заночует. Ему 
даже матрац купили специальный, только он его не признавал. <<К чему, "--- 
говорит, "--- такие роскоши?>>. И спихнет матрац с подоконника. А потом всю 
ночь вертится, спать не дает.

Гоголь переоделся Пушкиным, пришел к Пушкину и позвонил. Пушкин открыл ему и 
кричит: <<Смотри, Арина Родионовна, я пришел!>>.

Лермонтов хотел у Пушкина жену увести. На Кавказ. Все смотрел на нее из-за
колонн, смотрел\dots Вдруг устыдился своих желаний. <<Пушкин, "--- думает, "---
зеркало русской революции, а я? свинья>>. Пошел, встал перед ним на колени и 
говорит: <<Пушкин, где твой кинжал? Вот грудь моя>>. Пушкин очень смеялся.

Однажды Пушкин стрелялся с Гоголем. Пушкин говорит:

"--* Стреляй первым ты.

"--* Как я? Нет, ты.

"--* Ах, я! Нет, ты!

Так и не стали стреляться.

Однажды Пушкин решил испугать Тургенева и спрятался на Тверском бульваре под 
лавкой. А Гоголь тоже решил в этот день испугать Тургенева, переоделся Пушкиным 
и спрятался под другой лавкой. Тут Тургенев идет. Как они оба выскочат!\dots

Однажды Пушкин написал письмо Рабиндранату Тагору. <<Дорогой далекий друг, "---
писал он, "--- я Вас не знаю, и Вы меня не знаете. Очень хотелось бы 
познакомиться. Всего хорошего. Саша>>.

Когда письмо принесли, Тагор предавался самосозерцанию. Так погрузился, хоть 
режь его. Жена толкала, толкала, письмо подсовывала "--- не видит. Он, правда, 
по"=русски читать не умел. Так и не познакомились.

Однажды Гоголь шел по Тверскому бульвару (в своем виде) и встретил Пушкина. 
<<Здравствуй, Пушкин, "--- говорит, "--- что ты все стихи да стихи пишешь? Давай
вместе прозу напишем>>. <<Прозой только\dots\dots\dots\dots хорошо>>, "--- 
возразил Пушкин.