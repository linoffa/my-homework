Однажды Федору Михайловичу Достоевскому, царствие ему небесное, исполнилось 150
лет. Он очень обрадовался и устроил день рождения. Пришли к нему все писатели, 
только почему"=то все наголо обритые. У одного Гоголя усы нарисованы. Ну хорошо,
выпили, закусили, поздравили новорожденного, царствие ему небесное, сели играть 
в вист. Сдал Лев Толстой "--- у каждого по пять тузов. Что за черт? Так не 
бывает. <<Сдай"=ка, брат Пушкин, лучше ты>>. <<Я,  "--* говорит, "--* 
пожалуйста, сдам>>. И сдал. У каждого по шесть тузов и по две пиковые дамы. Ну 
и дела\dots <<Сдай"=ка ты, брат Гоголь>>. Гоголь сдал\dots Ну, знаете\dots 
Даже и нехорошо сказать\dots Как"=то получилось так\dots Нет, право, лучше не 
надо.

Однажды Федор Михайлович Достоевский, царствие ему небесное, сидел у окна и 
курил. Докурил и выбросил окурок из окна. Под окном у него была керосиновая 
лавка. И окурок угодил как раз в бидон с керосином. Пламя, конечно, столбом. В  
одну ночь пол"=Петербурга сгорело. Ну, посадили его, конечно. Отсидел, вышел, 
идет в первый же день по Петербургу, навстречу "--- Петрашевский. Ничего ему не 
сказал, только пожал руку и в глаза посмотрел. Со значением.

Однажды у Достоевского засорилась ноздря. Стал продувать "--- лопнула перепонка 
в ухе. Заткнул пробкой "--- оказалась велика, череп треснул\dots Связал 
веревочкой "--- смотрит, рот не открывается. Тут он проснулся в недоумении, 
царствие ему небесное.

Федор Михайлович Достоевский страстно любил жизнь, царствие ему небесное. Она 
его, однако, не баловала, поэтому он часто грустил. Те же, кому жизнь улыбалась 
(например, Лев Толстой) не ценили это, постоянно отвлекаясь на другие предметы. 
Например, Лев Толстой очень любил детей. Они же его боялись. Они прятались от 
него под лавку и шушукались там: <<Робя, вы этого бойтесь "--- еще как трахнет 
костылем!>>. Дети любили Пушкина. Они говорили: <<Он веселый. Смешной такой>>. И 
гонялись за ним стайкой. Но Пушкину было не до детей. Он любил один дом на 
Тверском бульваре, одно окно в этом доме. Он мог часами сидеть на широком 
подоконнике, пить чай, смотреть на бульвар. Однажды, направляясь к этому дому, 
он поднял глаза и на своем окне увидел\dots себя. С бакенбардами, с перстнем на 
большом пальце. Он, конечно, понял, кто это. А вы?