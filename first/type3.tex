Лев Толстой очень любил детей. Однажды он играл с ними весь день и проголодался.
<<Сонечка, "--* говорит, "--* а, ангелочек, сделай мне тюрьку>>. Она возражает: 
<<Левушка, ты же видишь, я ``Войну и мир'' переписываю>>. <<А"=а"=а, "--* 
возопил он, "--* так я и знал, что тебе мой литературный фимиам дороже моего 
``Я''>>. И костыль задрожал в его судорожной руке.

Лев Толстой очень любил детей. Однажды он шел по Тверскому бульвару и увидел 
впереди Пушкина. <<Конечно, это уже не ребенок, это уже подросток, "--* подумал
Лев Толстой, "--* все равно, дай догоню и поглажу по головке>>. И побежал 
догонять Пушкина. Пушкин же, не зная толстовских намерений, бросился наутек. 
Пробегая мимо городового, сей страж порядка был возмущен неприличной быстротою 
бега в людном месте и бегом устремился вслед с целью остановить. Западная пресса
потом писала, что в России литераторы подвергаются преследованиям со стороны 
властей.

Лев Толстой очень любил детей. Утром проснется, поймает кого-нибудь и гладит по 
головке, пока не позовут завтракать.

Лев Толстой очень любил детей. Бывало, приведет в кабинет штук шесть, всех 
оделяет. И надо же: вечно Герцену не везло "--- то вшивый достанется, то 
кусачий. А попробуй поморщиться "--- хватит костылем.

Лев Толстой жил на площади Пушкина, а Герцен "--- у Никитских ворот. Обоим по 
литературным делам часто приходилось бывать на Тверском быльваре. И уж если 
встретятся "--- беда: погонится Лев Толстой и хоть раз, да врежет костылем по
башке. А бывало и так, что впятером оттаскивали, а Герцена из фонтана водой в
чувство приводили. Вот почему Пушкин к Вяземскому"=то в гости ходил, на окошке  
сидел. Так этот дом потом и назвался "--- дом Герцена.

Лев Толстой очень любил играть на балалайке (и, конечно, детей), но не умел. 
Бывало, пишет роман <<Война и мир>>, а сам думает: 
<<Тень"=дер"=день"=тер"=тер"=день"=день"=день>>. Или: 
<<Брам"=пам"=дам"=дарарам"=пам"=пам>>.